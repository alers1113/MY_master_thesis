\documentclass[a4paper, fontsize=14pt]{scrartcl}
\usepackage{cmap} % сразу после \documentclass
\usepackage[left=3cm,right=1.5cm,top=2cm,bottom=2cm]{geometry}
\usepackage{indentfirst}
\setlength{\parindent}{1.25cm} % для шрифта 14pt
\setcounter{page}{2} % нумерация страниц начнётся с 2
\usepackage{setspace}
\onehalfspacing
\textheight=24cm % высота текста

\usepackage[]{cite}
\usepackage[T2A]{fontenc}
\usepackage[utf8]{inputenc}
\usepackage[english, russian]{babel}
\usepackage{amsmath, amsfonts,amssymb}
\usepackage{graphicx, epsfig}
\graphicspath{{pictures/}}
\DeclareGraphicsExtensions{.pdf,.png,.jpg}
\usepackage{subfig}
\usepackage{color}
\usepackage{gensymb}
 
\usepackage{enumitem}	
\setlist{nolistsep} 



\newcommand\argmin{\mathop{\arg\min}}
\newcommand{\T}{^{\text{\tiny\sffamily\upshape\mdseries T}}}
\newcommand{\hchi}{\hat{\boldsymbol{\chi}}}
\newcommand{\hphi}{\hat{\boldsymbol{\varphi}}}
\newcommand{\bchi}{\boldsymbol{\chi}}
\newcommand{\A}{\mathcal{A}}
\newcommand{\B}{\mathcal{B}}
\newcommand{\x}{\mathbf{x}}
\newcommand{\hx}{\hat{x}}
\newcommand{\hy}{\hat{y}}
\newcommand{\M}{\mathcal{M}}
\newcommand{\N}{\mathcal{N}}
\newcommand{\R}{\mathbb{R}}
\newcommand{\p}{p(\cdot)}
\newcommand{\q}{q(\cdot)}
\newcommand{\uu}{\mathbf{u}}
\newcommand{\vv}{\mathbf{v}}

\usepackage{amsmath}
\usepackage{amsthm}

\usepackage{algorithm}
\usepackage{algpseudocode}
\floatname{algorithm}{Алгоритм}
\algnewcommand{\IIf}[1]{\State\algorithmicif\ #1\ \algorithmicthen}
\algnewcommand{\EndIIf}{\unskip\ \algorithmicend\ \algorithmicif}
\renewcommand{\algorithmicend}{\textbf{завершим}}
\renewcommand{\algorithmicif}{\textbf{если}}
\renewcommand{\algorithmicelse}{\textbf{иначе}}
\renewcommand{\algorithmicthen}{\textbf{то}}
\renewcommand{\algorithmicdo}{\textbf{выполним}}
\renewcommand{\algorithmicfor}{\textbf{цикл для}}
\renewcommand{\algorithmicforall}{\textbf{цикл для всех}}
\renewcommand{\algorithmicwhile}{\textbf{цикл пока}}




\usepackage{hyperref}
\usepackage [section] {placeins}


\begin{document}


\addcontentsline{toc}{section}{\protect\numberline{}Аннотация}
\section*{Аннотация}
\begin{abstract}

--текст-- 
  
  \bigskip
  \textbf{Ключевые слова}: \emph{климат, радиационный форсинг, черный углерод, альбедо снега, климатическая модель, радиационная модель}
\end{abstract}


\newpage
\phantomsection
\addcontentsline{toc}{section}{\protect\numberline{}Содержание}
\tableofcontents


\newpage
\section*{Введение}
\addcontentsline{toc}{section}{\protect\numberline{}Введение}

В течении последних десятилетий вопросы, связанные с изменениями климата, стали особенно актуальными. Для предсказания будущих изменений активно используются климатические модели. Они представляют собой модели общей циркуляции атмосфры и океана. С ростом вычислительных мощностей модели постоянно усложняются и уточняются. Одной из важнейших переменных в таких моделях является альбедо. Данная величина характеризует отражающие свойства поверхности.

Целью данной работы было получение полноценной модели альбедо заснеженной поверхности, которую можно было бы как использовать для отдельных локальных расчетов, так и внедрить в глобальную климатическую модель, например, INMCM5 \cite{Volodin2017}.  

В текущей версии климатической модели ИВМ РАН реализована простая параметризация альбедо, зависящая только от температуры поверхности, поэтому использование более подробного описания данной величина способно уточнить модель. 

Другим актуальным приложением модели альбедо является задача расчета радиационного форсинга от загрязнения примесями поверхности Зели. Выпадая на снег, они изменяют его оптические свойства, что приводит к дополнительному нагреву планеты. Оценку этого нагрева можно получить, рассматривая изменение альбедо. 

Побочным результатом данной работы стала модернизация почвенно-снежного блока климатической модели ИВМ РАН. Был уточнено описание таяния снега, а также реализован процесс перезамерзания талой воды. Данные изменения способствуют более точному описанию переходных сезонов.

Отдельные результаты данной работы изложены в статье \cite{Chernenkov2021rus} и сборниках докладов  \cite{MSARD2019, mipt2019, EGU2020poster, EGU2020, mipt2020, Arctic2020}, а также докладывались и обсуждались на следующих научных конференциях, в том числе международных:
\begin{itemize}
    \item Международный симпозиум «Атмосферная радиация и динамика» (МСАРД–2019), 25 – 27 июня 2019, Санкт-Петербург. \cite{MSARD2019}
    \item 62-я Всероссийская научная конференция МФТИ, секция ФПМИ – Вычислительные технологии и моделирование, Москва, 2019. \cite{mipt2019}
    \item EGU2020: Sharing Geoscience Online, ITS2.15/BG2.25 Pan-Eurasian Experiment (PEEX) – Observation, Modelling and Assessment in the Arctic-Boreal Domain, May, 2020. \cite{EGU2020poster, EGU2020} \sloppy 
    \item 63-я Всероссийская научная конференция МФТИ, Секция ФПМИ – Вычислительные технологии и моделирование, Москва, 2020. \cite{mipt2020}
    \item Вторая Всероссийская научная конференция «Мониторинг состояния и загрязнения окружающей среды. Экосистемы и климат Арктической зоны». Москва, 25-27 ноября 2020г. \cite{Arctic2020}
\end{itemize} 



\newpage
\section{Постановка задачи}

Целью данной работы было построение полноценной модели альбедо заснеженной поверхности, которая учитывала бы его зависимость от основных параметров, а также была бы пригодна для внедрения в глобальную климатическую модель, например, INMCM5 \cite{Volodin2017}. 

На первом этапе было необходимо определить основные факторы, которые оказывают влияние на альбедо. Таковыми оказались: структура снега, а именно форма и размеры снежных кристаллов, а также наличие в нем примесей и солнечное склонение.

Следующим шагом было получение зависимостей, описывающих характеристики, от которых зависит альбедо. Для этого было нужно построить:
\begin{itemize}
    \item технологию расчета концентраций примесей, содержащихся в снегу
    \item зависимость, описывающую изменение солнечного зенитного угла в течении астрономического года
    \item модель, описывающуюю процесс старения снега 
\end{itemize} 

В ходе изучения метаморфизма снега выяснилось, что в климтической модели ИВМ не хватает переменных для описания данного процесса. Поэтому возникла еще одна подзадача в виде модификации модели климата. Было необходимо ввести переменную, которая описывает содержание жидкой воды в слое снега, а также реализовать процессы перезамерзания талой воды и изменения плотности снега с течением времени.

В целом, результатов, полученных на предыдущих этапах, уже было достаточно для проведения расчетов альбедо по данным климатичской модели ИВМ РАН, например, с помощью радиационной модели SNICAR \cite{Flanner2007}. Однако, в таком случае расчет альбедо возможен только на этапе обработки данных, а хотелось бы иметь возможность проводить такие расчеты на каждом шаге по времени внутри модели. Внедрение модели SNICAR в климатическую модель ИВМ сопряжено со сложностями различного характера, поэтому следующим этапом стало построение параметризации альбедо на основании данных из модели SNICAR. 

Заключительным шагом было применение построенной модели альбедо для оценки радиационного форсинга от загрязнения снега черным углеродом. Таким образом, данная модель прошла тестирование на реальной задаче. 


\newpage
\section{Факторы, влияющие на альбедо снега}
Альбедо - это физическая величина, равная отношению отраженной от поверхности радиации к падающей. Например, для свежевыпавшего сухого снега альбедо составляет $0.85-0.95$, для лежалого загрязненного снега уже $0.40-0.50$, для морского льда $0.30-0.40$, а для древесного угля примерно $0.04$. Данная величина играет важную роль в описании климата. Ее уменьшение приводит к увеличению поглащения поверхностью приходящей радиации, что влечет за собой дополнительный нагрев планеты. 

Основными факторами, влияющими на отражающую способность снега являются форма и размеры снежных кристаллов, наличие различных примесей в снегу, в частности, черного углерода. Также важно отметить, что альбедо земной поверхности зависит от угла, под которым на поверхность приходит солнечная рдиация.


\subsection{Солнечный зенитный угол}

Значение солнечного склонения меняется в течении года из-за движения Земли по эллиптической орбите вокруг Солнца и наклона ее собственной оси вращения. Так, он равен нулю два раза в год: в дни весеннего и осеннего равноденствия. Из-за данного явления меняется солнечный зенитный угол, под которым на поверхность Земли приходит излучение, что непосредственно влияет на величину альбедо. Косинус зенитного угла для северного полушария выражается через угол склонения Солнца и широту местности следующим образом: пусть $\delta$ –- угол склонения Солнца, $\phi$ –- широта, $\theta$ –- солнечный зенитный угол, тогда:
\begin{equation}
    \cos \theta = \cos ( \phi - \delta ) \label{sys}
\end{equation}
при этом:
\begin{equation}
    \sin \delta = \sin \varepsilon \sin \left(\dfrac{2 \pi (d - d_e)}{365} \right) ,  \label{sys}
\end{equation}
где $\varepsilon$ = 23.45$\degree$  – наклон Земли к плоскости эклиптики, $d$ – время (в сутках с начала года), $d_e$ = 80.5.


\begin{figure}[h]
    \begin{minipage}[h]{0.49\linewidth}
        \center{\includegraphics[width=1.1\linewidth]{coszen2.png} \\ (а)}
    \end{minipage}
    \hfill
    \begin{minipage}[h]{0.49\linewidth}
        \center{\includegraphics[width=1.1\linewidth]{coszen3.png} \\ (б)}
    \end{minipage}
    \caption{Изменение (а) косинуса солнечного зенитного угла и (б) альбедо снега в течении года (при остальных фиксированных услоиях), вызванные вращением Земли вокруг Солнца (на примере точки, находящейся на $70\degree$ С.Ш.)}
    \label{fig:image}
\end{figure}


\subsection{Метаморфизм снега}

С течением времени после снегопада из-за изменений температуры, а также под действием давления снежные кристаллы изменяют свою форму от изначально близкой к сферической до очень сложных (например, фрактальная "Снежинка Коха"), а также увеличиваются в размерах, слипаясь \cite{Grenfell1999, Grenfell2005, He2018}. Это приводит к уменьшению отражающей способности заснеженной поверхности, что особенно заметно в случае длин волн, относящихся к ближнему ИК-диапазону ($0.7-5.0$ мкм). Также важным фактором для описания метаморфизма является наличие в слое снега жидкой воды и перезамерзших кристаллов. Эти факторы играют наибольшую роль в переходные сезоны, когда происходит активное таяние или, ноборот, формирование снежного покрова.

\begin{figure}[h]
    \begin{minipage}[h]{0.5\linewidth}
        \center{\includegraphics[width=1\linewidth]{rds3.png} \\ (а)}
    \end{minipage}
    \hfill
    \begin{minipage}[h]{0.5\linewidth}
        \center{\includegraphics[width=1\linewidth]{rds2.png} \\ (б)}
    \end{minipage}
    \caption{(а) Спектральное альбедо, соответствующие различным формама и размерам снежного кристалла и (б) зависимость альбедо от эффективного радиуса снежного кристалла}
    \label{fig:image}
\end{figure}

\newpage
Для описания снежных кристаллов с достаточно хорошой точностью можно применять приближение, использующее ледяные шарообразные частицы, сохраняющие удельную площадь поверхности ($SSA$) (в том числе, в случаях кристаллов сложных форм) \cite{Grenfell1999}. Их основная характеристика -- эффективный радиус ($r_e$). Он определяется как средневзвешенный по площади радиус ансамбля таких частиц и связан с удельной площадью поверхности ($SSA$) и плотностью льда $\rho_{ice}$ следующим образом \cite{Flanner2006}:  
\begin{equation}
    r_e = \dfrac{3} {\rho_{ice} \cdot SSA} \label{sys}
\end{equation}
Таким образом, старение снега можно описывать как эволюцию эффективного радиуса снежного кристалла. Рассмотрим зависимость, использованную в климатической модели CLM4.5 \cite{CLM4.5tech}. Пусть момент времени $t$ соответствует текущему шагу по времени, а $(t - 1)$ -- предыдущему. Тогда старение снега можно описать следующим уравнением:
\begin{equation}
    r_e(t) = [r_e (t - 1) + \delta r_{e , dry} + \delta r_{e , wet} ] \cdot f_{old} + r_{e ,0} \cdot f_{new} + r_{e , rfz} \cdot f_{rfrz}, \label{sys}
\end{equation}
где $ r_{e ,0} = 54.5 $ мкм и $r_{e , rfz} = 1000 $ мкм -- значения эффективного радиуса, соответствующие свежевыпавшему и перезамерзшему снегу, $\delta r_{e , dry}$ и $\delta r_{e , wet}$ -- вклады от метаморфизма сухого и мокрого снега, $f_{old}$, $f_{new}$ и $f_{rfrz}$ -- доли лежалого, свежевыпавшего и перезмерзшего снега соответственно.  

Эволюция сухого снега описывается следующим уравнением \cite{Flanner2007, CLM4.5tech}:
\begin{equation}
    \dfrac{d}{dt} \delta r_{e , dry} = {\left( \dfrac{dr_{e , dry}}{dt} \right)}_0 \left(\dfrac{\eta}{\eta + (r_e - r_{e, 0})}\right)^{1 / \kappa}, \label{sys}
\end{equation}
где ${\left( \dfrac{dr_{e , dry}}{dt} \right)}_0$, $\eta$, $\kappa$ -- некоторые табличные параметры, зависящие от температуры снега, температурного градиента и плотности снега, причем для каждого значения $SSA$ они различные. Эти данные находятся в открытом доступе (\url{http://snow.engin.umich.edu/snowaging/}). Они охватывают следующие дипазоны параметров: по температуре от 223.15 до 273.15 К, по плотности от 50 до 400 кг/м$^3$, по температурному градиенту от 0 до 300 К/м для следущего набора значений $SSA$: 60, 80, 100 м$^2$/кг. В работах \cite{CLM4.5tech, Flanner2006, Flanner2007} предлагается считать $SSA = 60$ м$^2$/кг.

Эволюция мокрого снега описывается следующим уравнением \cite{CLM4.5tech}:
\begin{equation}
\dfrac{d}{dt} \delta r_{e , wet} = \dfrac{10^{18} C^3 f_{liq}} {4 \pi r_{e}^2}, \label{sys}
\end{equation}
где $C = 4.22 \cdot 10^{-13}$, $f_{liq}$ - доля жидкой воды в снегу.

Учитывая, что нас интересует суммарный вклад и от сухого, и от мокрого снега, предыдущие два уравнения можно объединить в одно. Используя в качестве начального условия для эффективного радиуса значение с прошлого шага по времени, получаем следующую задачу Коши:
\begin{equation}
    \begin{cases}
        \dfrac{d}{dt} \delta (r_{e , dry} + r_{e , wet}) = {\left( \dfrac{dr_{e , dry}}{dt} \right)}_0 \left(\dfrac{\eta}{\eta + (r_e - r_{e, 0})}\right)^{1 / \kappa} + \dfrac{10^{18} C^3 f_{liq}} {4 \pi r_{e}^2} ,
        \\
        \delta (r_{e , dry} + r_{e , wet})(t-1) = 0
    \end{cases} \label{sys}
\end{equation}
Данное дифферненциальное уравнение предлагается решать численно.

\begin{figure}[h]
    \center{\includegraphics[scale=0.85]{rds1.png}}
    \caption{Эволюция эффективного радиуса снежного кристалла за первые 2 недели января, рассчитанное по описанной выше зависимости}
    \label{fig:image}
\end{figure}
 

\subsection{Загрязнение снега атмосферными аэрозолями}

Выпадение на заснеженную поверхность атмосферных аэрозолей приводит к загрязнению и, как следствие,  уменьшаению ее отражающей способности. Одним из таких аэрозолей является черный углерод (ЧУ) или сажа. Основными источниками сажевого аэрозоля являются выбросы, возникающие при сжигании различных видов топлива, а также лесные и степные пожары. В качестве меры загрязнения можно рассматривать концентрацию черного углерода в снегу.

\begin{figure}[h]
    \center{\includegraphics[scale=0.9]{alb_cbc.png}}
    \caption{Зависимость снежного альбедо от концентрации ЧУ в снегу}
    \label{fig:image}
\end{figure}

Пусть имеются следующие среднемесячные климатические данные:
\begin{itemize}
    \item $H_{snw}$ -- водно-эквивалентная толщина снега, 
    \item $I_{bc}$ -- поток черного углерода на поверхности снежного слоя, 
    \item $Q_{melt}$ -- поток талой воды на нижней границе снежного слоя,
    \item $\sigma$ -- доля ячейки сетки, покрытая снегом,
\end{itemize} 

\newpage
Также дополнительно предположим, что имеет место равномерное перемешивание частиц атмосферного аэрозоля и снежных гранул.

Рассмотрим следующие промежуточные поля за n-ый месяц, необходимые для вычисления концентрации ЧУ в снегу: 
\begin{itemize}
    \item приток массы ЧУ в ячейке, покрытой снегом:
    \begin{equation}
        P_{bc}^n = I_{bc}^n \Delta t \sigma^n S   \label{sys}
    \end{equation}
    \item масса снега в ячейке сетки:
    \begin{equation}
        M_{sn}^n = H_{snw}^n \rho_w S   \label{sys}
    \end{equation}
\end{itemize} 
Здесь $\rho_w$ = 1000 кг/м$^3$ - плотность воды, $S$ –- площадь ячейки сетки, $\Delta t$ = 1 месяц.
    
Массу ЧУ в заснеженной ячейке сетки $M_{bc}$ можно рассчитать на основе балансового соотношения, предложенного в работе \cite{Flanner2007}:

\begin{equation}
    \dfrac{d M_{bc}}{d t} = - C_{MSE} ~ Q_{melt} ~ \dfrac{M_{bc}}{M_{sn}} ~ S + I_{bc} ~ \sigma S     \label{sys}
\end{equation}
Данное соотношение можно переписать в дискретном виде следующим образом:

\begin{equation}
   M_{bc}^{n+1} = M_{bc}^n - C_{MSE} ~ Q_{melt}^{n+1} ~ \dfrac{M_{bc}^n}{M_{sn}^n} ~ S \Delta t + P_{bc}^{n+1}     \label{sysBalance}
\end{equation}
Здесь $Q_{melt} = - \dfrac{1}{S} \dfrac{dM_{sn}}{dt} \geq 0$ --  средний по ячейке поток массы растаявшего снега, $C_{MSE}$ - коэффициент вымывания частиц ЧУ талой водой. Если считать, что поток ЧУ из атмосферы $I_{bc}^{n + 1} = 0$, то из прошлого уравнения следует, что:

\begin{equation}
   C_{MSE} = \dfrac{\Delta M_{bc} / M_{bc}}{\Delta M_{sn} / M_{sn}}     \label{sys}
\end{equation}
То есть коэффициент $C_{MSE}$ –- это отношение вымываемой относительной массы ЧУ к относительной
массе талой воды. Для гидрофильного черного углерода $C_{MSE} = 0.2$, для гидрофобного $C_{MSE} = 0.03$ \cite{Flanner2007, Conway1996}. В расчетах с временными масштабами порядка месяца с хорошей точность можно считать весь черный углерод гидрофильным, так как характерное время его перехода в такое состояние из гидрофобного составляет порядка 2 дней.

В качестве начальных данных для дискретного балансового уравнения \eqref{sysBalance} можно использовать следующее выражение:
\begin{equation}
    M_{bc}^0 = P_{bc}^0, \label{sysStart}
\end{equation}
где нулевой индекс означает месяц, когда в данной ячейке сетки появился снежный покров. 

Зная решение задачи \eqref{sysBalance}, \eqref{sysStart} за каждый месяц, можно найти среднемесячную концентрацию ЧУ в снегу по следующей формуле:
\begin{equation}
   C_{bc}^n = \dfrac{M_{bc}^n}{M_{sn}^n}  \label{sys}
\end{equation}

\begin{figure}[h]
    \center{\includegraphics[scale=0.6]{Cbc1.jpg}}
    \caption{Среднемесячная концентрация черного углерода в снегу за январь 1998 год, рассчитанная по данным климатической модели ИВМ РАН, [нг/г]}
    \label{fig:image}
\end{figure}

Таким образом, получена технология для расчета концентрации примесей в снегу. Причем, ее можно исользовать как на этапе обработки климатических данных, так и внутри глобальной модели.

\newpage
\section{Модификация глобальной климатической модели ИВМ РАН}

\subsection{Общие сведения о климатической модели ИВМ РАН}

Рассматривается пятая версия модели INMCM5 \cite{Volodin2017}. Модель климата ИВМ РАН состоит из двух основных блоков: модели общей циркуляции атмосферы и модели общей циркуляции океана. Она имеет разрешение по долготе-широте $2 \times 1.5\degree$ в атмосфере и $0.5 \times 0.25\degree$ в океане. По высоте в атмосфере она содержит 71 уровень, а океане - 40 уровней. 

Атмосферная блок основан на стандартной системе уравнений гидротермодинамики с гидростатическим приближением, записанной в адвективной форме. Прогностические переменные модели - горизонтальные составляющие ветра, температура, удельная влажность и приземное давление. Для аппроксимации по пространству используются конечные разности второго порядка на С-сетке Аракавы. Для интегрирования по времени используется схема "чехарда".

Данная версия модели участвует в международном проекте по сравнению климатических моделей CMIP6 (Coupled Models Intercomparison Project). Некоторые результаты по моделированию современного климата с помощью данной модели представлены в работе \cite{Volodin2017}.  \sloppy 

Стоит отметить, что в данной версии модели для описания альбедо заснеженной поверхности на суше используется следующая параметризация, зависящая от температуры поверхности $T$:
\begin{equation}
    \alpha_{sn} = \begin{cases}
                        0.8,~ T \leq -1 ~\degree \mathrm{C}; \\
                        0.8 - 0.1(T + 1), ~T > -1 ~\degree \mathrm{C};
                  \end{cases} \label{sys}
\end{equation}
В сравнении с другими моделями альбедо из модели INMCM5 несколько занижено, например, \cite{Flanner2007, Gueymard2019}.

\newpage
\begin{figure}[h]
    \center{\includegraphics[scale=0.5]{Albedo_01_1997_old.png}}
    \caption{Среднемесячное альбедо (январь 1997 г.) по данным модели INMCM5}
    \label{fig:image}
\end{figure}

\subsection{Описание почвенно-снежного блока модели}

Необходимая модификация модели для более точного расчета альбедо снега затрагивает почвенно-снежный блок, который входит в атмосферный блок, поэтому сначала опишем его. 

Введем необходимые переменные:
\begin{itemize}
    \item $t$ -- время, 
    \item $z$ -- вертикальная координата, направленная вниз, 
    \item $T$ -- температура, 
    \item $W$ -- количество жидкой влаги в долях от количества льда, 
    \item $V$ -- количество водяного пара, 
    \item $I$ -- количество льда, 
    \item $\lambda_T$ -- коэффициент теплопроводности, 
    \item $\lambda_W$ и $\lambda_V$ -- коэффициенты диффузии воды и водяного пара, 
    \item $\delta$ -- коэффициент термовлагопроводности из-за  температурного градиента, 
    \item $\rho$ -- плотность почвы, 
    \item $C$ -- теплопроводность почвы, 
    \item $\gamma$ -- скорость инфильтрации воды под действием силы тяжести, 
    \item $L_i$ -- удельная теплота плавления льда, 
    \item $L_v$ -- удельная теплота парообразования, 
    \item $F_i$ -- скорость изменения количества жидкой влаги и льда при замерзании (таянии), 
    \item $F_v$ -- скорость изменения содержания водяного пара и воды при испарении, 
    \item $R_f$ -- изменение влагосодержания из-за горизонтального стока воды, 
    \item $R_r$ -- скорость всасывания воды корневой системой растительности,
\end{itemize} 

В климатической модели ИВМ РАН данный блок представлен одномерной моделью, которая описывает процессы тепло- и влагопереноса в почве, растительности и снежном покрове, а также обмен этой системы теплом и влагой с атмосферой \cite{Volodin1998, Volodina2000}. Основой данной модели является численное решение следующей системы уравнений:

\begin{equation}
    \rho ~ C \dfrac{\partial T}{\partial t} = \dfrac{\partial }{\partial z} \lambda_T \dfrac{\partial T}{\partial z} + L_i F_i - L_v F_v  \label{sys1}
\end{equation}
\begin{equation}
    \dfrac{\partial W}{\partial t} = \dfrac{\partial }{\partial z} \lambda_W \left( \dfrac{\partial W}{\partial z} + \delta \dfrac{\partial T}{\partial z} \right) + \dfrac{\partial \gamma}{\partial z} - F_i - F_v - R_f - R_r  \label{sys2}
\end{equation}
\begin{equation}
    \dfrac{\partial V}{\partial t} = \dfrac{\partial }{\partial z} \lambda_V \dfrac{\partial V}{\partial z} + F_v  \label{sys3}
\end{equation}
\begin{equation}
    \dfrac{\partial I}{\partial t} = F_i  \label{sys4}
\end{equation}
Уравнения $\eqref{sys1} - \eqref{sys4}$ решаются в слое $(0, H)$, где $H$ соответствует горизонту в почве, на котором отсутствуют внутрисезонные изменения погоды.

Если поверхность почвы покрыта слоем снега толщиной $h$, то для описания процесса телопереноса в слое $(-h, 0)$ используется следующее уравнение:
\begin{equation}
    \rho_{sn} C_{sn} \dfrac{\partial T_{sn}}{\partial t} = \dfrac{\partial }{\partial z} \lambda_{sn} \dfrac{\partial T_{sn}}{\partial z},  \label{sys5}
\end{equation}
где $T_{sn}, \rho_{sn}, C_{sn}, \lambda_{sn}$ - температура, плотность, теплоемкость и теплопроводность снега.

Для задания граничных условий системы $\eqref{sys1} - \eqref{sys3}, \eqref{sys5}$ сверху, если почва покрыта снегом, то на границе $z = -h$, а если нет, на $z = 0$, считаются заданными температура подстилающей поверхности, количество водяного пара в воздухе и поток жидкой влаги, обусловленный дождевыми осадкаими, таянием снега и испарением с поверхности почвы. В свою очередь, температура подстилающей поверхности находится из уравнения теплового баланса подстилающей поверхности. На нижней границе $z = H$ задаются климатическое распределение температуры почвы и отсутствие диффузионных потоков воды и пара. 

\subsection{Описание внедренных модификаций}

Для совместного использования модели альбедо, учитывающей старение снега, и глобальной климатической модели ИВМ были необходимы следующие модификации. Во-первых, нужно было добавить в модель такую переменную, как доля жидкой воды в слое снега, а также реализовать возможность перезамерзания талой воды. Во-вторых, было необходимо добавить эволюцию плотности снега, так как ранее она считалась постоянной: $\rho_{snow} = 0.1854 $ г/cм$^3$.

Определим переменные, необходимые для описания внесенных изменений: 
\begin{itemize}
    \item $S$ -- водно-эквивалентная толщина слоя снега,
    \item $S_{sn}$ -- "настоящий"\  снег (по сути, крошка из пористого льда), 
    \item $M_{soil}$ -- вода, поступившая на поверхность почвы,
    \item $P$ -- интенсивность осадков при температуре подстилающей поверхности, меньшей $0 ~\degree$C,
    \item $\lambda$ -- удельная теплота плавления, 
    \item $\mathcal{L}$ -- удельная теплота парообразования, 
    \item $\mathcal{E}$ -- поток скрытого тепла на поверхность снега, идущего на испарение/сублимацию,
    \item $\rho_w$ -- плотность воды (1 г/см$^3$),
    \item $M$ -- интенсивность снеготаяния,
    \item $S_{wat}$ -- талая вода, оставшаяся в слое снега,
    \item $F$ -- интенсивность замерзания воды,
    \item $S_{rfrz}$ -- перезамерзшая талая вода (по сути, крошка из сплошного льда),
    \item $T_{sn}$ -- температура снега, 
    \item $\Delta E$ -- избыточный/дефицитный поток тепла в тепловом балансе на поверхности.
\end{itemize}

Обычная высота снежного покрова $h$ связана с водно-эквивалентной высотой $S$ следующим соотношением:
\begin{equation}
    \rho_w S =  \int\limits_{-h}^0 \rho_{sn} \, dz\  \label{sys}
\end{equation}

В текущей версии модели водно-эквивалентная толщина слоя снега вычисляется на основании следующего уравнения баланса \cite{Volodin1998, Volodina2000}:
\begin{equation}
    \dfrac{\partial S}{\partial t} = P - M - \dfrac{\mathcal{E}}{\mathcal{L} \cdot \rho_w}  \label{sys}
\end{equation}
Процесс таяния начинается, когда температура подстилающей поверхности становится больше $0 ~\degree$C, при этом весь растаявший снег, а также возможный дождь, выпавший при наличии снежного покрова, поступают на поверхность почвы.

Идея модификации заключается в более физичном описании процесса таяния снега. Так, теперь предполагается, что при таянии снега вода не уходит моментально на верхнюю границу почвы, а постепенно просачивается сквозь толщу снега. При этом талая вода может замерзать, отдавая тепло снежному покрову. Стоит отметить, что теперь снежный пкров представляется как некоторая субстанция, состоящия из трех фракций: непосредственно снег, талая вода, содержащаяся в нем и перезамерзший снег, который, по сути, является ледяной крошкой. Данный процесс реализуется следующим алгоритмом:

\begin{algorithm}[H]
\caption{Процессы таяния снега и перезамерзания талой воды}
\label{alg:setup}
\begin{algorithmic}[]
    \If{$ \Delta E >0$, $T_{sn} \geq 0 ~\degree$C и $S^{n-1} > 0 $}
        \State $ M = \dfrac{\Delta E}{\lambda} $ 
        \State $ \delta = \dfrac{S_{sn}^{n-1}}{S_{sn}^{n-1} + S_{rfrz}^{n - 1}}$ 
        \State $ S_{sn}^n = S_{sn}^{n-1} + P - \Delta t \left( \dfrac{\mathcal{E}}{\mathcal{L} \cdot \rho_w} + \delta \cdot M \right) $ 
        \State $ S_{rfrz}^n = S_{rfrz}^{n - 1} - \Delta t (1 - \delta)M $ 
        \State $ S_{wat}^{max} = f( S_{sn}^n ) $
        \State $ \Delta S_{wat} = max\{\Delta t \cdot M, ~S_{wat}^{max}\} $ 
        \State $ M_{soil} = max\{\Delta t \cdot M - S_{wat}^{max}, ~0\} $ 
        \State $ S_{wat}^n = S_{wat}^{n-1} + \Delta S_{wat} $ 
    \Else
        \If{$S^{n-1} > 0$, $S_{wat}^{n-1} > 0$ и $T_{sn} \leq 0 ~\degree$C}
            \State $F = -\dfrac{\Delta E}{\lambda}$
            \State $S_{sn}^n = S_{sn}^{n-1} + P - \Delta t \left( \dfrac{\mathcal{E}}{\mathcal{L} \cdot \rho_w} - F \right)$
            \State $S_{wat}^n = max\{ S_{wat}^{n-1} - \Delta t \cdot F, ~0\}$
            \State $S_{rfrz}^n = S_{rfrz}^{n - 1} + ( S_{wat}^{n-1} - S_{wat}^n )$
            
        \EndIf
    \EndIf
    \State $S^n = S_{sn}^n + S_{wat}^n + S_{rfrz}^n$
\end{algorithmic}
\end{algorithm}

Критическая масса воды $S_{wat}^{max}$, способная содержаться в слое снега, с одной стороны, является фунцией от количества снега, с другой стороны, определяется его пористостью $\varepsilon_{sn}$. Предлагается оценивать ее по следующей формуле \cite{Gusev2002}:
\begin{equation}
     S_{wat}^{max} = S ~\dfrac{\varepsilon_{sn}}{1 - \varepsilon_{sn}}  \label{sys}  
\end{equation}
Пористость снега, в свою очередь, связана с его плотностью \cite{Stock}:
\begin{equation}
    \varepsilon_{sn} = 0.11 \left( \dfrac{\rho_w}{\rho_{sn}} - 1 \right)  \label{sys}  
\end{equation}
Эволюцию плотности снега предлагается описывать аналогично тому, как это сделано в модели SWAP \cite{Gusev2002}, \cite{YOSIDA1955}:
\begin{equation}
    \rho_{sn}(\tau_i) = \rho_{sn}(\tau_{i-1}) \left[  1 + 0.1 H_{sn}(\tau_{i-1}) \exp \{ 0.08 T_{sn} - 21 \rho_{sn}(\tau_{i-1})  \} \right]    \label{sysRHOOLD}  
\end{equation}
В данной модели $H_{sn}$ -- водно-эквивалентная толщина слоя снега в сантиметрах, плотность снега $\rho_{sn}$ вычисляется в г/см$^3$, температура слоя снега $T_{sn}$ задается в градусах Цельсия. Нужно заметить, что шаг по времени здесь $\tau_{i} - \tau_{i-1} = 1$ сутки, поэтому для использования данной зависимости в модели ИВМ необходима переинтерполяция.

При этом, стоит учитывать, что слой снега может состоять как из уже лежалого снега, так и из свежевыпашего. Также на среднюю по всему слою плотность влияют талая вода, содержащаяся в нем, и перезамерзший снег. Чтобы учесть все эти факторы, предлагается рассчитывать плотнось снежного слоя как среднее взвешенное по все этим фракциям:
\begin{equation}
    \rho_{snow} = \rho_{old} \cdot \delta_{old} + \rho_{new} \cdot \delta_{new} + \rho_{w} \cdot \delta_{wat} + \rho_{ice} \cdot \delta_{rfrz}
\end{equation}
Здесь $\rho_{old}$ -- плотность уже лежалого снега, рассчитанная с использованием формулы (\ref{sysRHOOLD}),  $\rho_{new} = 0.1$ г/см$^3$ -- плотность свежевыпавшего снега, $\rho_{w} = 1$ г/см$^3$ -- плотность воды, $\rho_{ice} = 0.917$ г/см$^3$ -- плотность льда, $\delta_{old}, ~\delta_{new}, ~\delta_{wat}, ~\delta_{rfrz}$ -- массовые доли (в водном эквиваленте) старого и нового снега, талой воды, а также перезамерзшего снега в снежном слое соответственно.


\subsection{Результаты внедрения модификаций}

Для тестирования внесенных изменений были проведены расчеты климата с 1997 по 2002 года с исходной и модифицированной версиями модели. Из результатов расчетов следует, что учет влагосодержания снега и реализация процесса перезамерзания талой воды приводят к тому, что снег сходит на месяц позже, а процесс формирования снежного покрова происходит более интенсивно. Так, теперь в северном полушарии в Заполярье, в горах на западном побережье Канады и Аляски, а также в районе Гималаев в некоторых районах снег продолжает сохраняться даже до конца июня, в то время как раньше он практически весь успевал стаять в течении мая. Это хорошо согласуется с архивными данными наблюдений за климатом, например, данные National Centers for Environmental Information (NCEI). Вклад описанных процессов в формирование устойчивого снежного покрова в конце осени - начале зимы наиболее заметен в заполярных регионах Евразии и Гималаях. Это ожидаемо, так как процесс перезмерзания реализуется в переходные сезоны, когда температура колеблется около нулевой отметки.

\begin{figure}[h]
    \begin{minipage}[h]{0.49\linewidth}
        \center{\includegraphics[width=1.1\linewidth]{Snow_depth_05_ave_1997-2002_default.png} \\ (а)}
    \end{minipage}
    \hfill
    \begin{minipage}[h]{0.49\linewidth}
        \center{\includegraphics[width=1.1\linewidth]{Snow_depth_05_ave_1997-2002_new.png} \\ (б)}
    \end{minipage}
    \caption{Среднемесячная водно-эквивалентная толщина слоя снега по данным (а) исходной и (б) модифицированной версий модели ИВМ (месяц -- май) }
    \label{fig:image}
\end{figure}

Также нужно отметить, что более подробное описание почвенно- снежного блока привело к увеличению количества снега в целом (Рис. \ref{fig:imageSn}). Наибольшее различие наблюдается в весенние месяцы, при этом с июля по сентябрь различий практически нет. Это сравнение показывает, что реализованные процессы дают ощутимый вклад в образование снежного покрова, но вместе с тем они не создают каких-либо некорректных аномалий в летний период.  В сравнении с данными наблюдений суммарная масса снега в случае модифицированной версии оказывается заметно завышенной, но, вместе с тем, описание площади, покрытой снегом, наоборот, улучшается. Можно отметить, что данный результат согласуется с тенденцией к завышению массы снега при более точном описании площади покрытия в ряде климатических моделей, участвующих в CMIP6, которые используют более подробное описании снега \cite{Mudryk2020}.

\begin{figure}[h]
    \begin{minipage}[h]{0.48\linewidth}
        \center{\includegraphics[width=1.1\linewidth]{snow_mass.png} \\ (а) }
    \end{minipage}
    \hfill
    \begin{minipage}[h]{0.51\linewidth}
        \center{\includegraphics[width=1.1\linewidth]{snow_square.png} \\ (б) }
    \end{minipage}
    \caption{(а) Годовой ход массы снега и (б) площади, покрытой им, осредненные за 1997-2002 годы по данным исходной и модифицированной версиий модели}
    \label{fig:imageSn}
\end{figure}

Другим важным результатом внесенных изменений в модель стала возможность проводить расчеты снежного альбедо с учетом метаморфизма снега, вызванного его старением, так как были добавлены недостающие модельные переменные.


\newpage
\section{Модель альбедо}

Полученные выше реультаты позволяют перейти к вычислению альбедо. Для для вычисления данной величины можно использовать как уже готовую модель, учитывающую описанные ранее факторы, так и некоторую параметризацию. 

\subsection{Описание радиационной модели SNICAR}
В качестве примера модели альбедо можно рассматривать локально-одномерную радиационную модель SNICAR (SNow-ICe-AErosole radiation model) \cite{Flanner2007}. Данная модель описывает вертикальный перенос излучения в слое снега с заданными профилями концентраций содержащихся в нем атмосферных аэрозолей, плотности среды и размеров снежных гранул, а одним из ее выходных данных является спектральное альбедо заснеженной поверхности. Данную модель можно использовать, например, для расчета альбедо на этапе обработки данных из климатической модели. Внедрение данной модели в глобальную модель климата представляется достаточно сложной задачей. Однако, на ее основании можно построить параметризацию зависимости альбедо от основных параметров и уже ее внедрять в модель. Это может позволить проводить расчеты альбедо непосредственно в процессе работы глобальной модели и потоенциально уточнить ее.

В любом случае, используя модель SNICAR или параметризацию, построенную на ее основе, замыкаясь данными из модифицированной климатической модели ИВМ РАН, мы получаем полноценную модель альбедо.

\subsection{Построение параметризации альбедо на основе модели SNICAR}

Будем строить параметризацию альбедо от 3 переменных: эффективного радиуса снежного кристалла $r_e$, концентрации черного углерода в слое снега $C_{bc}$ и косинуса солнечного зенитного угла $coszen = cos(\theta)$ ($\theta$ -- солнечный зенитный угол). Предлагается строть параметризацию в два этапа: сначала получить зависимость от $r_e$ и $C_{bc}$, полагая $coszen = 1$, а затем уточить ее за счет учета косинуса зенитного угла.

На первом этапе предлагается искать параметризацию альбедо в виде полинома от двух переменных:
\begin{equation}
    alb_1(r_e, C_{bc}) = \sum_{i,j = 0}^N \sigma_{i.j} r_e^i C_{bc}^j   \label{sys}  
\end{equation}
Эмперически установленно, что наилучший результат получается в случае $N = 5$.

На втором этапе альбедо уточняется на основании эмперической зависимости \cite{Saito2019}:
\begin{equation}
    alb(r_e, C_{bc}, coszen) = \alpha + (A + B \cdot \alpha^C) \left( \dfrac{1 - coszen}{1 + coszen} \right)^D     \label{sys}  
\end{equation}
Здесь $\alpha = alb_1(r_e, C_{bc})$ - приближение с первого шага.

Параметры $\{ \sigma_{i.j} \}, A, B, C, D$ находятся на основании данных, полученных из результатов запусков модели SNICAR с различными значениями эффективного радиуса снежного кристалла, концентрации черного углерода и косинуса солнечного зенитного угла.

Полученная параметризация альбедо достаточно проста с вычислительной точки зрения, что делает возможным в будущем использование ее в глобальной климатической модели на каждом шаге по времени.


\newpage
\section{Применение модели альбедо для вычисления радиационного форсинга, вызванного загрязнением заснеженной поверхности}

Климат Земли и его чувствительность к различным воздействиям определяются естественными и антропогенными изменениями радиационного баланса Земли - радиационным форсингом. Выпадая на снег, атмосферные аэрозоли уменьшают альбедо поверхности, что создает дополнительный радиационный форсинг. Это может приводить к более быстрому таянию снега и повышению приземной температуры в весенний сезон. 

В данной главе оценивается форсинг, вызванный загрязнением снега черным углеродом. Дополнительный нагрев предлагается вычислять на основании изменения альбедо и данных о приходящей на поверхность радиации. Используя данные из климатической модели, проведены расчеты с помощью модели альбедо с учетом загрязнения снега черным углеродом и без.

Пусть $\alpha_{0}$ и $\alpha_{soot}$ - альбедо чистого и загрязненного снега соответственно. Пусть $\alpha^{vis}$ и $\alpha^{nir}$ - значения альбедо для видимого ($0.3-0.7$ мкм) и ближнего ИК ($0.7-5.0$ мкм) диапазонов длин волн падающего излучения. Тогда форсинг можно оценить по следующей формуле:
\begin{equation}
    R = \sigma [ (\alpha_{0}^{vis} - \alpha_{soot}^{vis})F_{down}^{vis} + (\alpha_{0}^{nir} - \alpha_{soot}^{nir})F_{down}^{nir} ] \label{sysFORC}  
\end{equation}
Здесь $\sigma$ - доля области, покрытая снегом, а потоки радиации определяются следующим образом: $F_{down}^{vis} \approx F_{down}^{nir} \approx 0.5 \cdot F_{sw}$, где $F_{sw}$ - поток приходящей коротковолновой радиации.

В формуле (\ref{sysFORC}) фигурируют альбедо и потоки радиации, только соответствующие видимому и ближнему инфракрасному диапазонам. Это оправдано, так как основной вклад в изменение спектрального альбедо из-за загрязнения снега приходится как раз на такие длины волн.

\begin{figure}[h]
    \center{\includegraphics[scale=0.6]{forcing_1997.png}}
    \caption{Среднегодовой радиационный форсинг из-за загрязнения снега черным углеродом, рассчитанный с помощь построенной модели альбедо (1997 г.)}
    \label{fig:image}
\end{figure}

Нужно отметить, что учет структуры снега также оказывает влияние на радиационный форсинг. В случае старого снега форсинг от осадков черного углерода заметно выше, чем если бы весь снег считался свежевыпавшим. Такая добавка может составлять локально до $0.3$ Вт/м$^2$. 



%%%%%%%%%%%%%%%%%%%%%%%%%%%%%%%%%%%%%%%%%%%%%%%%%%%%%%%%%%%%%%%%%%%%%%%%%%%%%%%%%%%%%%%%%%%%%%%%%%%%%%%%%%%%%%%%%%%%%%%%%%%%%%%%%%%%%%%
\newpage
\section*{Заключение}
\addcontentsline{toc}{section}{\protect\numberline{}Заключение}






%%%%%%%%%%%%%%%%%%%%%%%%%%%%%%%%%%%%%%%%%%%%%%%%%%%%%%%%%%%%%%%%%%%%%%%%%
\newpage
\phantomsection
\addcontentsline{toc}{section}{\protect\numberline{}Список литературы}
\bibliographystyle{gost2008}
\bibliography{biblio}







\end{document} 